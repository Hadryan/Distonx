\documentclass[a4paper,12pt]{article}
\usepackage{mathtext}
\usepackage[english,russian]{babel}
\usepackage[T2A]{fontenc}	
\usepackage{amssymb} 
\usepackage{amsmath}
\usepackage[utf8]{inputenc}	
\usepackage[unicode, pdftex]{hyperref}
\usepackage{xcolor}
\definecolor{linkcolor}{HTML}{799B03}
\definecolor{urlcolor}{HTML}{799B03} 
\makeatletter
\renewcommand*\env@matrix[1][*\c@MaxMatrixCols c]{%
	\hskip -\arraycolsep
	\let\@ifnextchar\new@ifnextchar
	\array{#1}}
\makeatother
\usepackage[center]{titlesec}

\begin{document}
\subsection*{Общее:}
	
\href{https://github.com/binance-exchange/binance-official-api-docs/blob/master/web-socket-streams.md}{Ссылка на информацию о всех доступных данных.}

Если что, в паре BTC/USDT quote asset - это USDT, base asset - это BTC. Market Taker - тот, кто выставляет рыночный ордер.

\textbf{Пары:} BTC/USDT, ETH/USDT, BCH/USDT, BNB/USDT, LTC/USDT, ETH/BTC, BCH/BTC, BNB/BTC, LTC/BTC, BCH/BNB, LTC/BNB. Таким образом, всего имеем пять крипт: биткоин, эфириум, bitcoin cash, лайткоин и binance coin, а также, собственно, доллар (точнее, USD Tenther). Пар всего 11.

\textbf{Это не те данные, которые мы непосредственно будем запихивать в нейросеть! Это те данные, которые мы получаем через api, а их обработка - это еще отдельная тема.}

\subsection*{Для каждой пары:}
\begin{itemize}
	\item \textbf{<symbol>@aggTrade и <symbol>@trade:} \\
	Эти потоки в реальном времени кидают информацию о сделках. Здесь считаем следующие статистики для 1 минуты и 1 секунды: mean, median, max, min, std, sum. Это производим для вот каких переменных:
	\begin{itemize}
		\item $BaseVolume$ - объемы продаж в base asset за соответствующую единицу времени.
		\item $QuoteVolume$ - объем продаж за соответствующую единицу времени в quote asset.
		\item $Price$ - цена.
	\end{itemize}
	Итого - 36 переменных ($2 \cdot 6 \cdot 3$).
	А также:
	\begin{itemize}
		\item $MinuteNumber/SecondNumber$ - количество сделок за единицу времени.
		\item $SecondOpenPrice/MinuteOpenPrice$ - цена первой сделки.
		\item $LastPrice$ - последняя цена.
		\item $SecondTakerPercent/MinuteTakerPercent$ - процент сделок, где покупателем выступает Market Taker.
	\end{itemize}
	\textbf{Всего 43 поля.}
	\item \textbf{<symbol>@kline\_<interval>:} \\
	<interval> - это 1m, 3m, 5m, 15m, 30m, 1h, 2h, 4h, 6h, 8h, 12h. Поток возвращает разные метрики цены за соответствующий интервал. Для каждого интервала:
	\begin{itemize}
		\item $TimeDelta$ - время с последнего обновления.
		\item $TradeNumber$ - число сделок.
		\item $OpenPrice/ClosePrice/HighPrice/LowPrice$ - статистики цены.
		\item $BaseVolume$ - объем продаж в base asset.
		\item $QuoteVolume$ - объем продаж в quote asset.
		\item $TakerBaseVolume/TakerQuoteVolume$ - объем продаж среди сделок, когда покупателем выступает Market Taker.
	\end{itemize}
	\textbf{Всего 110 полей.}
	\item \textbf{<symbol>@ticker:}\\
	Этот поток возвращает статистику по паре о последних 24 часах.
	\begin{itemize}
		\item $PriceChange/PriceChangePercent$ - изменение цены.
		\item $WeightedAveragePrice$ - средняя цена.
		\item $FirstPrice/LastPrice$ - цена первой и последней сделок. 
		\item $LastQuantity$ - объем последней сделки.
		\item $BestBidPrice/BestAskPrice$ - цены лучшего бида и аска.
		\item $BestBidQuantity/BestAskQuantity$ - объемы лучшего бида и аска.
		\item $TradeNumber$ - число сделок.
	\end{itemize}
	\textbf{Всего 11 полей.}
	\item \textbf{<symbol>@depth<levels>@100ms:}\\
	Этот поток каждые 100ms кидает изменения ордербука. <levels> возьмем 20. Для каждого уровня:
	\begin{itemize}
		\item $BidPrice\text{<}level\text{>}/BidQuantity\text{<}level\text{>}$ - цена и объем ордера продажи, находящегося на определенном уровне ордербука.
		\item $AskPrice\text{<}level\text{>}/AskQuantity\text{<}level\text{>}$ - цена и объем ордеров покупки.
	\end{itemize}
	\textbf{Всего 80 полей.}
\end{itemize}

Итого 244 полей для каждой пары. Всего, получается, полей 2684 - и это за каждую секунду. Биг дата, чтоб ее!

\end{document}